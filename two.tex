%% Copyright 1998 Pepe Kubon
%%
%% `two.tex' --- 2nd chapter for thes-full.tex, thes-short-tex from
%%               the `csthesis' bundle
%%
%% You are allowed to distribute this file together with all files
%% mentioned in READ.ME.
%%
%% You are not allowed to modify its contents.
%%

%%%%%%%%%%%%%%%%%%%%%%%%%%%%%%%%%%%%%%%%%%%%%%%%%
%
%     Chapter 2   
%
%%%%%%%%%%%%%%%%%%%%%%%%%%%%%%%%%%%%%%%%%%%%%%%%

\chapter{Preliminaries}
In this project, we ...
\section{Technical Terms}
\subsection{Native-side}
\begin{description}
	\item[User] \hfill \\
	Users need to signup on the native site to have full access to functionalities on Social Needle. When signing up, users can choose to link his/her native account with Facebook or Twitter. In this way, all the online activities will have the unified credential across all three platforms. 
	Users can follow each other to get updates of their activities. 
	\item[Post] \hfill \\
	All discussions on the native site will be organized into posts. A post has a title, its content, a set of tags and comments. Every user can create a post to initiate any discussion. A post on the native site corresponds to a ``tweet'' on Twitter and a ``status'' on Facebook Page. All content will be synchronized to Facebook and Twitter.
	\item[Tag] \hfill \\
	The author of a post can add one or more tags to the post when it is created. Users can follow tags and thus get updates on all content with the tags. 
	\item[Collection] \hfill \\
	Users can create collections to group posts into meaningful set. Unlike tags which belongs to a public tag pool, collections belong to individual users. Personal collections are visible to everyone, so user can follow others' collections and get updates of any posts in the collections. Users can create filter to import new posts with certain tags to their collections; or they can add posts to collections manually. 
	\item[News Feeds] \hfill \\
	News feeds is the list of posts on the homepage when user signs in. It is consisted of posts of followed users, followed tags and followed collections. 
\end{description}
\subsection{Facebook-side}
\begin{description}
	\item[Facebook] \hfill \\
	Facebook is one of the most popular SNSs on the Internet. As of January 2011, the number of active users on Facebook has reached 600 million \cite{facebook_600m}. On Facebook, users can create their own profiles and easily communicate with their friends. Facebook provides two ways for users to ``get together'' in cyberspace: Group and Page, both enabling users to come together online to share information and discuss specific topics. In spite of that most Group's functions overlap with Page's, we choose to use Page mainly because Page comes with an administrator that can post on Page's wall under the name of the Page. For example, on a Page called ``CMPT307'' the administrator can post under the user name ``CMPT307''. This ``authoritative voice'' facilitates the broadcasting of information in scenarios like conference and courses. Page supports posting text, photos, and videos, commenting on posts, organizing events, and many other customized functions. In our project, we focus on building a prototype supporting only textual post and comment.
	\begin{description}
		\item[User] \hfill \\
		a person who has an account on Facebook
		\item[Page Administrator] \hfill \\
		a special user who creates a Facebook Page
		\item[Page] \hfill \\
		a Facebook Page that has a wall for both administrator and users to post
		\item[Post] \hfill \\
		 a Status administrator or users post to the Page's wall
		\item[Comment] \hfill \\ 
		 a comment on the Status
	\end{description}
	\item[Facebook Graph API] \hfill \\ 
	Facebook provides Graph API to allow developers to read and write to Facebook. With Graph API, developers can query anything about a person or a Page on Facebook as long as they are authenticated through the OAuth protocol. They can post things (e.g. text, photos, videos) to a person's wall or a Page through the Graph API as well. Graph API provides a simple and consistent view of Facebook's social graph, uniformly representing objects (e.g., people, photos, events, and pages) and the connections between them (e.g., friendships, likes, and photo tags)\footnote{http://developers.facebook.com/docs/reference/api/}.
\end{description}

\subsection{Twitter-side}

\begin{description} 
\item[Twitter] \hfill \\
Twitter is a web site that offers social networking and
  microblogging website service, enabling its user to send and read
  messages called \textit{tweets}~\cite{twitter_wiki}. Because of its
  simplicity and easy of use, Twitter has gain exceptional popularity.
  People use it to share emotions, opinions, news, etc.  A lot of
  companies, organizations or famous people are now ``on Twitter'',
  which means they update status on a Twitter account.  For example,
  Simon Fraser University (SFU) has a twitter account @SFU that
  constantly updates news and other information about the university.

  With the popularity of Twitter, terms have been invented for concepts
in Twitter over the years. Some of these term are used heavily in this
report, so they are explained in following part.

\item[Tweet] \hfill \\
 A tweet is a status posted on Twitter. It can
  have no more than 140 characters. Tweet can also be used as a verb
  to mean the action of posting a status.

\item[Follow] \hfill \\
 Following someone on Twitter means you are
  subscribing to their tweets as a follower, and that you will see
  their updates in your \textit{home timeline}~\cite{follow}. The
  concept of timeline is covered next.

\item[Timeline] \hfill \\
 A timeline is a collected stream of tweets
  listed in a real-time order~\cite{timeline}. There are different
  types timelines. The two mostly used timelines are \textit{home
    timeline} and \textit{user timeline}. The home timeline is the one
  you see when you log in to Twitter, which lists all the tweets by
  yourself and the users you follow. The user timeline is the one you
  see when you click on a Twitter user's profile, which lists all the
  tweets from this specific user.

\item[Reply] \hfill \\
A reply is any tweet posted by clicking the
 ``Reply'' button on another tweet ~\cite{mention_reply}. The text of a
  reply always begin with ``@username'', where username is the name of
  the user who tweet you reply to.

\item[Mention] \hfill \\
 A mention of someone is any tweet that
  contains ``@username'' anywhere in the body of the tweet, where
  username is the name of the user mentioned. By this definition,
  replies are also considered mentions. You can see all the mentions
  of you by clicking on the @Mentions tab on your home page.

\item[Retweet] \hfill \\
A retweet is a repeat of another’s tweet under
  your own account without the need to copy and paste~\cite{morris09}.
  Retweets provides an easy way for the users to share each other's
  status. Retweet can also mean the action of posting a repeat of
  another tweet.

\end{description}

\section{Technologies}

\subsection{Ruby on Rails} % (fold)
\label{sub:ruby_on_rails}

Ruby on Rails is an open source web application framework for Ruby. It uses the Model-View-Controller (MVC) architecture for application development.

% subsection ruby_on_rails (end)
\subsection{Etc.}


