%% Copyright 1998 Pepe Kubon
%%
%% `two.tex' --- 2nd chapter for thes-full.tex, thes-short-tex from
%%               the `csthesis' bundle
%%
%% You are allowed to distribute this file together with all files
%% mentioned in READ.ME.
%%
%% You are not allowed to modify its contents.
%%

%%%%%%%%%%%%%%%%%%%%%%%%%%%%%%%%%%%%%%%%%%%%%%%%%
%
%     Chapter 2   
%
%%%%%%%%%%%%%%%%%%%%%%%%%%%%%%%%%%%%%%%%%%%%%%%%

\chapter{Preliminaries}
In this project, we use various tools to ...
\section{Terminologies}
\subsection{Native-side}
\begin{description}
	\item[User] \hfill \\
	Users need to signup on the native site to have full access to functionalities on Social Needle. When signing up, users can choose to link his/her native account with Facebook or Twitter. In this way, all the online activities will have the unified credential across all three platforms. 
	Users can follow each other to get updates of their activities. 
	\item[Post] \hfill \\
	All discussions on the native site will be organized into posts. A post has a title, its content, a set of tags and comments. Every user can create a post to initiate any discussion. A post on the native site corresponds to a ``tweet'' on Twitter and a ``status'' on Facebook Page. All content will be synchronized to Facebook and Twitter.
	\item[Tag] \hfill \\
	The author of a post can add one or more tags to the post when it is created. Users can follow tags and thus get updates on all content with the tags. 
	\item[Collection] \hfill \\
	Users can create collections to group posts into meaningful set. Unlike tags which belongs to a public tag pool, collections belong to individual users. Personal collections are visible to everyone, so user can follow others' collections and get updates of any posts in the collections. Users can create filter to import new posts with certain tags to their collections; or they can add posts to collections manually. 
	\item[News Feeds] \hfill \\
	News feeds is the list of posts on the homepage when user signs in. It is consisted of posts of followed users, followed tags and followed collections. 
\end{description}
\subsection{Facebook-side}

\subsection{Twitter-side}

\section{Technologies}



