%% Copyright 1998 Pepe Kubon
%%
%% `two.tex' --- 2nd chapter for thes-full.tex, thes-short-tex from
%%               the `csthesis' bundle
%%
%% You are allowed to distribute this file together with all files
%% mentioned in READ.ME.
%%
%% You are not allowed to modify its contents.
%%

%%%%%%%%%%%%%%%%%%%%%%%%%%%%%%%%%%%%%%%%%%%%%%%%%
%
%     Chapter 2   
%
%%%%%%%%%%%%%%%%%%%%%%%%%%%%%%%%%%%%%%%%%%%%%%%%

\chapter{Preliminaries}
In this project, we use various tools to ...
\section{Technical Terms}
\subsection{Native-side}
\begin{description}
	\item[User] \hfill \\
	User is ...
	\item[Post] \hfill \\
	Post is ...
	\item[Tag] \hfill \\
	Tag is
	\item[Collection] \hfill \\
	Collection is ...
	\item[News Feeds] \hfill \\
	News feeds is ...
\end{description}
\subsection{Facebook-side}
\begin{description}
	\item[Facebook] \hfill \\
	Facebook is one of the most popular SNSs on the Internet. As of January 2011, the number of active users on Facebook has reached 600 million\footnote{http://www.businessinsider.com/facebook-has-more-than-600-million-users-goldman-tells-clients-2011-1}. On Facebook, users can create their own profiles and easily communicate with their friends. Facebook provides two ways for users to ``get together'' in cyberspace: Group and Page, both enabling users to come together online to share information and discuss specific topics. In spite of that most Group's functions overlap with Page's, we choose to use Page mainly because Page comes with an administrator that can post on Page's wall under the name of the Page. For example, on a Page called ``CMPT307'' the administrator can post under the user name ``CMPT307''. This ``authoritative voice'' facilitates the broadcasting of information in scenarios like conference and courses. Page supports posting text, photos, and videos, commenting on posts, organizing events, and many other customized functions. In our project, we focus on building a prototype supporting only textual post and comment.
	\begin{description}
		\item[User] \hfill \\
		a person who has an account on Facebook
		\item[Page Administrator] \hfill \\
		a special user who creates a Facebook Page
		\item[Page] \hfill \\
		a Facebook Page that has a wall for both administrator and users to post
		\item[Post] \hfill \\
		 a Status administrator or users post to the Page's wall
		\item[Comment] \hfill \\ 
		 a comment on the Status
	\end{description}
	\item[Facebook Graph API] \hfill \\ 
	Facebook provides Graph API to allow developers to read and write to Facebook. With Graph API, developers can query anything about a person or a Page on Facebook as long as they are authenticated through the OAuth protocol. They can post things (e.g. text, photos, videos) to a person's wall or a Page through the Graph API as well. Graph API provides a simple and consistent view of Facebook's social graph, uniformly representing objects (e.g., people, photos, events, and pages) and the connections between them (e.g., friendships, likes, and photo tags)\footnote{http://developers.facebook.com/docs/reference/api/}.
\end{description}

\subsection{Twitter-side}

\section{Technologies}



