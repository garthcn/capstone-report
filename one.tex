%% Copyright 1998 Pepe Kubon
%%
%% `one.tex' --- 1st chapter for thes-full.tex, thes-short-tex from
%%                the `csthesis' bundle
%%
%% You are allowed to distribute this file together with all files
%% mentioned in READ.ME.
%%
%% You are not allowed to modify its contents.
%%

%%%%%%%%%%%%%%%%%%%%%%%%%%%%%%%%%%%%%%%%%%%%%%%%%
%
%       Chapter 1 
%
%%%%%%%%%%%%%%%%%%%%%%%%%%%%%%%%%%%%%%%%%%%%%%%%

\chapter{Introduction}
Social Needle is a vertical social network service, which can be deployed in various domains. Each instantiation targets at a specific interests group and facilitates high-quality, frictionless, and easy-to-use communications between people interested in common topics. For example, students in a course can have their instantiation of Social Needle and discuss course-related content there; conference organizers can create their instantiation for their conference, and people attending that conference will be able to have domain-specific communications online. The philosophy behind Social Needle is to do one thing at a time and do it right. 

\section{Background}
Social network has become the hottest area in IT industry in recent years. With Facebook and Twitter becoming the third largest country on the earth with more than 600 million citizens, people around the world are putting a huge fraction of their social life online. 

However, Facebook plus Twitter is far from the end of the story for new social services. Though it seems that the power of Facebook and Twitter has reached to every corner of our social needs, they both have their weak points. Facebook failed to stimulate serious high-quality discussions, due to the distraction from its casual and largely entertaining content; Twitter's linear structure hurdles organized discussions and people have to resort to other sites (using short URLs) for more organized and meaningful discussion. Moreover, these problems are not likely to be addressed by both sites, since they stem from the basic nature and structure of each site.

\section{Social Needle}
Having a big picture of social network' landscape in mind, it is natural to come up with a brand new service, which provides dedicated and organized social network to specific interests groups. We call it Social Needle. 

As its name suggests, Social Needle aims to pinpoint the communication needs among a certain interests group. Unlike Facebook and Twitter which build an all-in-one social network for all your friends, Social Needle will be deployed for a group of people who share some common interests. For instance, organizers of an academic conference can utilize Social Needle to build a micro social network for this particular conference. All attendees and people interested in this conference can go there and have discussions on topics related to the conference. Clearly, Social Needle's domain-oriented design will be far more likely to spark professional and serious discussions than Facebook and Twitter. 

Besides, Social Needle will enable well-organized discussions such that people can easily find topics of their interests. It not only has tagging functionality for effortless search, but also adopts hierarchical structure for easy navigation. 

In addition to the native site, Social Needle also comes with social modules for Facebook and Twitter. Once the creators of the micro social network set up a Facebook Page or a Twitter account, they will be able to easily synchronize all content across all platforms. In this way, people will have full access to every piece of information no matter which platform they use. 

We believe that Social Needle's unique features will spark people's ideas through organized and well-structured flow of information. 

The rest of the report is organized as follows: ...