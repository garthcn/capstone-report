%% Copyright 1998 Pepe Kubon
%%
%% `one.tex' --- 1st chapter for thes-full.tex, thes-short-tex from
%%                the `csthesis' bundle
%%
%% You are allowed to distribute this file together with all files
%% mentioned in READ.ME.
%%
%% You are not allowed to modify its contents.
%%

%%%%%%%%%%%%%%%%%%%%%%%%%%%%%%%%%%%%%%%%%%%%%%%%%
%
%       Chapter 1 
%
%%%%%%%%%%%%%%%%%%%%%%%%%%%%%%%%%%%%%%%%%%%%%%%%

\chapter{Introduction}
In this project, we developed a vertical social network platform, Social Needle. It provides high-quality, frictionless, and easy-to-use communications between people interested in common topics. 

\section{Background}
Social network services (SNS) has become the hottest area in IT industry in recent years. With Facebook and Twitter becoming the third largest country on the earth with more than 600 million users, people around the world are putting a huge fraction of their social life online. With SNSs, it is very easy to find people with shared interests and form group with them. Not only users but also companies, conferences, and other organizations use the rich functionalities and social links provided by SNSs to create online groups for branding, discussion, and event organization.

However, Facebook plus Twitter is far from the end of the story for new social services. Though it seems that the power of Facebook and Twitter has reached to every corner of our social needs, they both have their weak points. Facebook failed to stimulate serious high-quality discussions, due to the distraction from its casual and largely entertaining content; Twitter's linear structure hurdles organized discussions and people have to resort to other sites (using short URLs) for more organized and meaningful discussion. Moreover, these problems are not likely to be addressed by both sites, since they stem from the basic nature and structure of each site.

In addition, as users begin to interact with different SNSs, the amount of uncategorized information will overwhelm them. They need to frequent each social network service for the latest information. At the same time, groups with the same shared interest may duplicate themselves in different SNSs. Members of those groups could hardly reach other members unless they are in the same SNS. In other words, current SNSs are isolated from one another, like islands in the sea \cite{10.1109/MIS.2008.50}. Users participate in discussion in one SNS cannot readily
access related discussions in other SNSs.

\section{Social Needle}
Having a big picture of social network's landscape in mind, it is natural to come up with a brand new service, which provides dedicated and organized social interactions to specific interests groups across various major social platforms. We call it Social Needle. 

As its name suggests, Social Needle aims to pinpoint the communication needs among a certain interests group. Unlike Facebook and Twitter which build an all-in-one social network for all your friends, Social Needle will be deployed for a group of people who share some common interests. For instance, organizers of an academic conference can utilize Social Needle to build a micro social network for this particular conference. All attendees and people interested in this conference can go there and have discussions on topics related to the conference. Clearly, Social Needle's domain-oriented design will be far more likely to spark professional and serious discussions than Facebook and Twitter. 

Besides, Social Needle will enable well-organized discussions such that people can easily find topics of their interests. It not only has tagging functionality for effortless search, but also adopts hierarchical structure for easy navigation. 

In addition to the native site, Social Needle also comes with social modules for other major SNSs. We choose to integrate with the two most popular SNSs: Facebook\footnote{http://www.facebook.com} and Twitter\footnote{http://www.twitter.com}. These two sites not only have the largest number of users but also are top choices for organizations and event organizers to share information. Once the creators of the their own social network set up a Facebook Page or a Twitter account, they will be able to easily synchronize all content across all platforms. In this way, people will have full access to every piece of information no matter which platform they use. 

We believe that Social Needle's unique features will spark people's ideas through free, organized and well-structured flow of information. 

\section{A Concrete Case: In-Course Discussions}
In the first phase of development, we choose to deploy Social Needle for in-course discussions at Simon Fraser University. This section will introduce the status quo of the in-course communications in university and how Social Needle will totally change the awkward workflows. 

Instructors and students in university may have noticed that the common ways for instructors to distribute information is to set up a course page or to use email. There are limitations of both approach. A course page is mostly static, and the information flows in only one direction: from an instructor to students. There is not too much interactions between instructors and students. On the other hand, communicating via emails can bring too many emails to instructor’s mailbox. Besides, the poor structure of email will quickly make it almost impossible to jump into discussion as the size of the email thread grows.

In addition to the communication between instructor an students, students also want to discuss problems with each other. Most students are likely to know only a few of the peer students in the class. When it comes to discussion, they tend to discuss only in small groups.

We attempt to address the above problems by introducing Social Needle. Instructors will be able to create courses on Social Needle and push all kinds of well-structure information to enrolled students. Students will be able to initiate and participate focused discussions related to the course and freely communicate with all members. Besides, the full integration with Facebook and Twitter will make all discussions accessible from all three platforms. Problem solved.   

The rest of the report will elaborate the functionalities and detailed implementations of Social Needle in the context of in-course discussion. The following parts are organized as follows: ...
