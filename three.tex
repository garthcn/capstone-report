%% Copyright 1998 Pepe Kubon
%%
%% `three.tex' --- 3rd chapter for thes-full.tex, thes-short-tex from
%%               the `csthesis' bundle
%%
%% You are allowed to distribute this file together with all files
%% mentioned in READ.ME.
%%
%% You are not allowed to modify its contents.
%%

%%%%%%%%%%%%%%%%%%%%%%%%%%%%%%%%%%%%%%%%%%%%%%%%%
%
%     Chapter 3   
%
%%%%%%%%%%%%%%%%%%%%%%%%%%%%%%%%%%%%%%%%%%%%%%%%

\chapter{User Stories}
The chapter will describe user stories on the native Social Needle site. The next chapter, Native-side Functionalities, will be based on this chapter. 

\section{Native-side}
\subsection{Instructors}
\begin{description}
	\item[As an instructor, I want to make administrative announcements.] \hfill \\
	The instructor logs in to the native site, goes to the course page. The instructor chooses to create a new post, fills in title, content, and tags the post as ``Announcement''. Simultaneously, the post will also be pushed to Facebook Page (created for the course) and the Twitter account (also created for the course). Students will be able to see the announcement from all three platforms.  
	\item[As an instructor, I want to answer questions posted by students.] \hfill \\
	The instructor sees a interesting question posted by a student. The instructor comments on the post and answers the question. His/Her answer will be pushed to Facebook Page and Twitter. 
	\item[As an instructor, I want to follow a post that interests me.] \hfill \\
	The instructor sees an interesting post. The instructor goes to the post viewing page and adds the post to one of his/her collections. Thus, the instructor will get all updates on the post in his/her news feeds.
\end{description}

\subsection{Students}
\begin{description}
	\item[As a student, I want to enrol in a course on Social Needle.] \hfill \\
	The student logs in to the native site and follows the course. 
	\item[As a student, I want to follow announcements made by instructor.] \hfill \\
	The student goes to the tags/collections page. The student sees the related announcement tag/collection for the course announcements, and follows the tags/collections. The student will get subsequent updates on topics on that course. 
	\item[As a student, I want to post my questions on Social Needle.] \hfill \\
	The student proceeds to create a new post. The student enters his/her questions, tags the question with appropriate tags and creates the post. It will appears in the news feeds of the students and instructors who follow the tags. In the meantime, the content of the post will be pushed to Facebook Page and Twitter, so that users can see the question from all three platforms.
	\item[As a student, I want to create a collection of the posts that interested me.] \hfill \\
	The student sees a interesting post and decide to keep it in his/her collection. In another words, the student thinks he/she may want to revisit the post later or just want to archive interesting topics. The student can create a collection in the post viewing page and add the current post to the collection. The student can also add the post to a existing collection by selecting the name of the collection. 
	\item[As a student, I want to follow a question.] \hfill \\
	The student sees an interesting post on the native site. The student goes to the post page and adds the post to one of his/her collections. Then the student will get updates on the posts in his/her news feeds.
	\item[As a student, I want to answer the questions posted by my classmates.] \hfill \\
	The student sees a interesting post in his/her news feeds. The student comments on the post and answers the question. All other students and instructors who follow this post will get updates about the new comments. Simultaneously, the answer will be pushed to Facebook Page and Twitter. On Twitter, the answer will be sent to all users who have retweeted or replied to the original post. 

\end{description}

\section{Facebook-side}
Facebook user story.
\section{Twitter-side}
Twitter user story.